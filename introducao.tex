\section*{Introdução}
    \addcontentsline{toc}{section}{Introdução}
        O decibel (dB) é uma medida da razão entre duas quantidades,
    sendo usado para uma grande variedade de medições em acústica,
    física, eletrônica e telecomunicações. Por ser uma razão entre duas
    quantidades iguais o decibel é uma unidade de medida adimensional
    semelhante a percentagem. O dB usa o logaritmo decimal (log10) para
    realizar a compressão de escala. Um exemplo típico de uso do dB é na
    medição do ganho/perda de potência em um sistema. Além do uso do
    dB como medida relativa, também existem outras aplicações na
    medidas de valores absolutos tais como potência e tensão entre outros
    (dBm, dBV, dBu). O emprego da subunidade dB é para facilitar o seu
    uso diário (Um decibel (dB) corresponde a um décimo de bel (B)). \cite{apostila_ifsc}
